\documentclass[12pt]{article}
\usepackage{amsmath,amssymb,amsfonts}
\usepackage{graphicx}
\usepackage[french]{babel}
\usepackage{tikz}
\usetikzlibrary{calc, arrows}
\usepackage{pgfplots}
\UseRawInputEncoding
\usepackage[utf8]{inputenc}
\usepackage[T1]{fontenc}
\usepackage{graphicx}
\usepackage{color}
\usepackage{listings}
\usepackage[a4paper, total={6in, 8in}]{geometry}
\usepackage{pdfpages}
\usepackage{enumitem}
\usepackage{array}
\usepackage{booktabs}
\usepackage{lastpage}
\usepackage{tcolorbox}
\usepackage{tikzpagenodes} % pour récupérer les dimensions de la page
\usepackage{adjustbox}
\usepackage{tabularx}
\usepackage{float}
\usepackage{array}
\usepackage{xcolor}

\usepackage{fancyhdr}

\pagestyle{fancy}
\fancyhf{}


\lhead{\textbf{La Bible du Réseaux en Licence}} % En-tête à gauche
\rhead{\textbf{2023-2024}} % En-tête à droite
\lfoot{\textbf{Despoullains Romain}} % Pied de page à gauche
\rfoot{\textbf{Page \thepage{} / \pageref*{LastPage}}} % Pied de page à droite
 
\definecolor{mygray}{rgb}{0.5,0.5,0.5}
\definecolor{mygreen}{rgb}{0,0.6,0}
\definecolor{myorange}{rgb}{1.0,0.4,0}

\lstset{
    language=Python,
    basicstyle=\footnotesize\ttfamily,
    keywordstyle=\color{mygreen},
    stringstyle=\color{myorange},
    commentstyle=\color{mygray},
    numbers=left,
    numberstyle=\tiny,
    stepnumber=1,
    numbersep=5pt,
    showspaces=false,
    showstringspaces=false,
    frame=single,
    breaklines=true,
    morecomment=[l]{\#}
}



\lstset{%
    language=Python,
    commentstyle=\color{mygray},
    morecomment=[l]{\#},
    basicstyle=\ttfamily\small
}

\title{La Bible du Réseaux en Licence}
\author{Despoullains Romain}
\date{}

\begin{document}
\begin{titlepage}
	\begin{center}
		\vspace*{1cm}
		
		\textbf{\LARGE INFO203 - INFO305}
		
		\vspace{0.5cm}
		{\Large Introduction aux réseaux informatiques}
		
		\vspace{2cm}
		
		\begin{tikzpicture}
        \clip (0,0) circle (0.25\textwidth);
        \node at (0,0) {\includegraphics[width=0.5\textwidth]{assets/1453228634-RobertKahn-3820539574.png}};
        \end{tikzpicture}
        
		\vspace{2cm}
		
		{\Huge \textbf{La Bible du Réseaux en Licence}}
		
		\vspace{1cm}
		
		{\Large \textbf{Despoullains Romain}}
		
		\vfill
		
		{\Large 2023-2024}
		
	\end{center}
\end{titlepage}

\tableofcontents
\pagebreak

\section{Introduction au monde TCP‐IP}

\subsection{Introduction}

\subsection{Glossaire pour le réseau}

\subsection{Concepts de l’interconnexion}

\section{Binaire et Héxadécimal}


\subsection{Binaire}


\subsection{Puissances de 2}

\begin{table}[h]
    \centering
    \begin{tabular}{|c|c|c|c|}
    \hline
    Puissance de 2 & Décimal & Binaire & Hexadécimal \\
    \hline
    $2^0$  & 1        & 1                & 1 \\
    \hline
    $2^1$  & 2        & 10               & 2 \\
    \hline
    $2^2$  & 4        & 100              & 4 \\
    \hline
    $2^3$  & 8        & 1000             & 8 \\
    \hline
    $2^4$  & 16       & 10000            & 10 \\
    \hline
    $2^5$  & 32       & 100000           & 20 \\
    \hline
    $2^6$  & 64       & 1000000          & 40 \\
    \hline
    $2^7$  & 128      & 10000000         & 80 \\
    \hline
    $2^8$  & 256      & 100000000        & 100 \\
    \hline
    $2^9$  & 512      & 1000000000       & 200 \\
    \hline
    $2^{10}$ & 1024   & 10000000000      & 400 \\
    \hline
    $2^{11}$ & 2048   & 100000000000     & 800 \\
    \hline
    $2^{12}$ & 4096   & 1000000000000    & 1000 \\
    \hline
    $2^{13}$ & 8192   & 10000000000000   & 2000 \\
    \hline
    $2^{14}$ & 16384  & 100000000000000  & 4000 \\
    \hline
    $2^{15}$ & 32768  & 1000000000000000 & 8000 \\
    \hline
    $2^{16}$ & 65536  & 10000000000000000& 10000 \\
    \hline
    \end{tabular}
    \caption{Puissances de 2 et leurs représentations en binaire et hexadécimal}
    \label{tab:puissances_de_2}
\end{table}

\subsection{Conversions entre décimal et binaire}

\subsection{Héxadécimal}


\subsection{Puissances de 16}

\begin{table}[h]
    \centering
    \begin{tabular}{|c|c|c|c|}
    \hline
    Puissance de 16 & Décimal & Binaire & Hexadécimal \\
    \hline
    $16^0$  & 1          & 1                 & 1 \\
    \hline
    $16^1$  & 16         & 10000             & 10 \\
    \hline
    $16^2$  & 256        & 100000000         & 100 \\
    \hline
    $16^3$  & 4096       & 1000000000000     & 1000 \\
    \hline
    $16^4$  & 65536      & 10000000000000000 & 10000 \\
    \hline
    \end{tabular}
    \caption{Puissances de 16 et leurs représentations en décimal, binaire et hexadécimal}
    \label{tab:puissances_de_16}
\end{table}

\begin{table}[h]
    \centering
    \begin{tabular}{|c|c|c|c|}
    \hline
    \(n \times 16\) & Décimal & Binaire & Hexadécimal \\
    \hline
    \(1 \times 16\) & 16  & 10000            & 10 \\
    \hline
    \(2 \times 16\) & 32  & 100000           & 20 \\
    \hline
    \(3 \times 16\) & 48  & 110000           & 30 \\
    \hline
    \(4 \times 16\) & 64  & 1000000          & 40 \\
    \hline
    \(5 \times 16\) & 80  & 1010000          & 50 \\
    \hline
    \(6 \times 16\) & 96  & 1100000          & 60 \\
    \hline
    \(7 \times 16\) & 112 & 1110000          & 70 \\
    \hline
    \(8 \times 16\) & 128 & 10000000         & 80 \\
    \hline
    \(9 \times 16\) & 144 & 10010000         & 90 \\
    \hline
    \(10 \times 16\) & 160 & 10100000        & A0 \\
    \hline
    \(11 \times 16\) & 176 & 10110000        & B0 \\
    \hline
    \(12 \times 16\) & 192 & 11000000        & C0 \\
    \hline
    \(13 \times 16\) & 208 & 11010000        & D0 \\
    \hline
    \(14 \times 16\) & 224 & 11100000        & E0 \\
    \hline
    \(15 \times 16\) & 240 & 11110000        & F0 \\
    \hline
    \(16 \times 16\) & 256 & 100000000       & 100 \\
    \hline
    \end{tabular}
    \caption{Table de multiplication par 16 en décimal, binaire et hexadécimal}
    \label{tab:table_de_16_formats}
\end{table}


\subsection{Conversions entre décimal et héxadécimal}


\subsection{Conversions entre binaire et héxadécimal}

\section{Utiles en Réseau}

\begin{table}[h]
    \centering
    \begin{tabular}{|c|c|c|}
    \hline
    Décimal & Binaire & Hexadécimal \\
    \hline
    16  & 00010000 & 10 \\
    \hline
    32  & 00100000 & 20 \\
    \hline
    48  & 00110000 & 30 \\
    \hline
    49  & 00110001 & 31 \\
    \hline
    52  & 00110100 & 34 \\
    \hline
    64  & 01000000 & 40 \\
    \hline
    80  & 01010000 & 50 \\
    \hline
    96  & 01100000 & 60 \\
    \hline
    98  & 01100010 & 62 \\
    \hline
    125 & 01111101 & 7D \\
    \hline
    128 & 10000000 & 80 \\
    \hline
    162 & 10100010 & A2 \\
    \hline
    168 & 10101000 & A8 \\
    \hline
    192 & 11000000 & C0 \\
    \hline
    201 & 11001001 & C9 \\
    \hline
    224 & 11100000 & E0 \\
    \hline
    240 & 11110000 & F0 \\
    \hline
    248 & 11111000 & F8 \\
    \hline
    252 & 11111100 & FC \\
    \hline
    254 & 11111110 & FE \\
    \hline
    255 & 11111111 & FF \\
    \hline
    \end{tabular}
    \caption{Conversion de nombres décimaux en binaire et hexadécimal}
    \label{tab:conversion}
\end{table}


\begin{table}[h]
    \centering
    \begin{tabular}{|c|c|c|}
    \hline
    Plage d'adresses & Description & Utilisation \\
    \hline
    0.0.0.0/8 & Actuelle réseau & Utilisé pour les broadcasts \\
    \hline
    10.0.0.0/8 & Privée & Utilisée dans les réseaux locaux \\
    \hline
    127.0.0.0/8 & Boucle locale & Pour les communications sur l'hôte local \\
    \hline
    169.254.0.0/16 & Lien local & Utilisée pour l'auto-configuration d'adresse \\
    \hline
    172.16.0.0/12 & Privée & Utilisée dans les réseaux locaux \\
    \hline
    192.0.0.0/24 & IETF Protocol Assignments & Réservée pour IETF Protocol Assignments \\
    \hline
    192.168.0.0/16 & Privée & Utilisée dans les réseaux locaux \\
    \hline
    198.18.0.0/15 & Test de performance & Utilisée pour des benchmarks \\
    \hline
    224.0.0.0/4 & Multicast & Réservée pour les adresses multicast \\
    \hline
    240.0.0.0/4 & Réservée & Réservée pour une utilisation future \\
    \hline
    255.255.255.255 & Broadcast & Adresse de broadcast limitée \\
    \hline
    \end{tabular}
    \caption{Quelques adresses IP et plages réservées et leurs utilisations}
    \label{tab:ip_reserved}
\end{table}


\begin{table}[h]
    \centering
    \begin{tabular}{|c|c|c|}
    \hline
    Préfixe CIDR & Masque de Sous-Réseau & Nombre d'Adresses IP \\
    \hline
    /32 & 255.255.255.255 & 1 \\
    \hline
    /31 & 255.255.255.254 & 2 \\
    \hline
    /30 & 255.255.255.252 & 4 \\
    \hline
    /29 & 255.255.255.248 & 8 \\
    \hline
    /28 & 255.255.255.240 & 16 \\
    \hline
    /27 & 255.255.255.224 & 32 \\
    \hline
    /26 & 255.255.255.192 & 64 \\
    \hline
    /25 & 255.255.255.128 & 128 \\
    \hline
    /24 & 255.255.255.0 & 256 \\
    \hline
    /23 & 255.255.254.0 & 512 \\
    \hline
    /22 & 255.255.252.0 & 1,024 \\
    \hline
    /21 & 255.255.248.0 & 2,048 \\
    \hline
    /20 & 255.255.240.0 & 4,096 \\
    \hline
    /19 & 255.255.224.0 & 8,192 \\
    \hline
    /18 & 255.255.192.0 & 16,384 \\
    \hline
    /17 & 255.255.128.0 & 32,768 \\
    \hline
    /16 & 255.255.0.0 & 65,536 \\
    \hline
    \end{tabular}
    \caption{Préfixes CIDR, masques de sous-réseau correspondants et nombre d'adresses IP}
    \label{tab:cidr}
\end{table}


\begin{table}[h]
    \centering
    \begin{tabular}{|c|c|c|c|}
    \hline
    Port & Protocole & Nom & Description \\
    \hline
    20 & TCP & FTP Data & Transfert de données FTP \\
    \hline
    21 & TCP & FTP Command & Commandes FTP \\
    \hline
    22 & TCP & SSH & Shell sécurisé \\
    \hline
    23 & TCP & Telnet & Connexion Telnet \\
    \hline
    25 & TCP & SMTP & Mail Transfer Agent (envoi de mails) \\
    \hline
    53 & UDP & DNS & Système de noms de domaine \\
    \hline
    80 & TCP & HTTP & Transfert hypertexte \\
    \hline
    110 & TCP & POP3 & Réception de mails \\
    \hline
    119 & TCP & NNTP & Réseau de transfert de nouvelles réseau \\
    \hline
    123 & UDP & NTP & Protocole de temps réseau \\
    \hline
    143 & TCP & IMAP & Accès aux messages Internet \\
    \hline
    443 & TCP & HTTPS & HTTP sur TLS/SSL \\
    \hline
    465 & TCP & SMTPS & SMTP sur TLS/SSL \\
    \hline
    587 & TCP & SMTP & Mail Submission Agent (soumission de mails) \\
    \hline
    993 & TCP & IMAPS & IMAP sur TLS/SSL \\
    \hline
    995 & TCP & POP3S & POP3 sur TLS/SSL \\
    \hline
    \end{tabular}
    \caption{Quelques ports communs et leurs utilisations}
    \label{tab:ports}
\end{table}


\begin{table}[h]
    \centering
    \begin{tabular}{|c|c|c|}
    \hline
    Numéro & Nom & Description \\
    \hline
    1 & ICMP & Internet Control Message Protocol \\
    \hline
    2 & IGMP & Internet Group Management Protocol \\
    \hline
    6 & TCP & Transmission Control Protocol \\
    \hline
    17 & UDP & User Datagram Protocol \\
    \hline
    41 & IPv6 & IPv6 encapsulation \\
    \hline
    47 & GRE & Generic Routing Encapsulation \\
    \hline
    50 & ESP & Encapsulating Security Payload \\
    \hline
    51 & AH & Authentication Header \\
    \hline
    88 & EIGRP & Enhanced Interior Gateway Routing Protocol \\
    \hline
    89 & OSPF & Open Shortest Path First \\
    \hline
    \end{tabular}
    \caption{Tableau de correspondance entre numéros de protocole et noms}
    \label{tab:protocol_numbers}
\end{table}


\section{Adressage IPV4}

\subsection{L’adressage Internet}

\subsection{Les classes d'adressage}

\section{Le protocole ARP}

\subsection{ARP}

\subsection{RARP}

\section{Le protocole IPV4}

\subsection{IPV4}

\subsection{Le datagramme IPV4}

\subsection{MTU}

\subsection{Routage des datagrammes}

\subsection{Le sous‐adressage}

\subsection{Le sous‐adressage cheatsheet}

\subsection{Le sous‐adressage variable}

\subsection{VLSM : agrégation de routes}

\section{Ethernet}

\subsection{La trame Ethernet}

\section{Le protocole UDP}

\subsection{User Datagram Protocol}

\subsection{Les ports}

\subsection{Format des messages}

\subsection{Pseudo en-tête}

\subsection{Multiplexage}

\subsection{Les ports standards}


\section{Le protocole ICMP}


\subsection{Internet Control Message Protocol}

\subsection{Format des messages}

\subsection{Format des commandes}

\subsection{Les commandes}

\subsection{Les messages d’erreur}

\subsection{Contrôle de congestion}

\subsection{Modification de route}

\subsection{Autres compte-rendus}

\section{Le protocole TCP}

\subsection{Transmission Control Protocol}

\subsection{La connexion}


\subsection{La Segmentation}


\subsection{L’Acquittement}


\subsection{Le fenêtrage}


\subsection{Gestion de la fenêtre}


\subsection{Structure du Segment}


\subsection{Format du segment}

\subsection{Mécanisme d’acquittement}


\subsection{Exemple}


\subsection{Mécanismes de retransmission}


\subsection{Gestion de la congestion}


\subsection{Mécanisme de connexion}


\subsection{Mécanisme de déconnexion}



\subsection{Ports standards}


\section{DNS}


\subsection{Domain Name System}


\subsection{La nécessité de nommer}


\subsection{Le principe pour l’utilisateur}


\subsection{L’envers du décor}


\subsection{Un système efficace}


\subsection{Les outils pour le DNS (ping -a, Nslookup, ipconfig /displaydns /flushdns}


\subsection{L’espace des noms de domaine}


\subsection{Les domaines}


\subsection{Les domaines de niveau supérieur}


\subsection{Le choix d’un nom de domaine}


\subsection{Lecture des noms}


\subsection{Les serveurs de noms}


\subsection{La délégation de zones}


\subsection{Types de serveurs de noms}


\subsection{Les résolveurs}


\subsection{Résoudre un nom}


\subsection{Les DNS Root servers}


\subsection{La résolution inverse}


\subsection{In-addr-arpa}


\subsection{Les enregistrements des DNS}


\subsection{Les logiciels DNS}


\subsection{Pourquoi installer un DNS}


\subsection{C’est quoi un nom de domaine ?}


\subsection{Domain Name Service}


\subsection{Domain Name Server}


\subsection{Serveur DNS sous Fedora (fichier hosts...)}


\subsection{Serveur DNS sous Fedora (configuration)}





\section{TCP/IP - Le routage dynamique}

\section{Le protocole IPV6}

\section{Adressage IPV6}

\section{WIFI (IEEE 802.11)}


\end{document}

